\chapter{Conclusion}

\textit{This dissertation has discussed the researching, implementation and evaluation of several semi-supervised models for use with genomic
data. It has also discussed techniques for processing data and improving model performance, and has used the lessons learned from 
this to construct a simple command line tool.}

\section{Success Criteria}

All the major success criteria of the project have been achieved. 

The implementations of the models performed as well as their original paper implementations on the MNIST dataset, showing that they were 
implemented correctly. All the models are now available online at https://github.com/Clondon98/Semi-Supervised\_Models, and include the 
best performing PyTorch implementation of the semi-supervised VAE and ladder that I was able to find online. This repository also allows 
the models to be easily used on different datasets, while other implementations are restricted to MNIST. 

The M2 and ladder model also succeeded in achieving statistically significant performance improvements over the fully supervised model on 
gene expression data, and an average of both of the models achieved even higher performance. This then led to the development of a 
command line tool that can be trained and used for classification very simply, without any knowledge of machine learning techniques being 
required.

I also managed to implement an extension that calculated the saliency of inputs to the network, and while time ran out to include it in the 
tool its results on MNIST showed that it was locating important features for the classification of the characters.

\section{Further Work}

Some areas of the project that could be explored further include:
\begin{itemize}
    \item Using VAEs to generate additional unlabelled datapoints and adding these to the dataset to see if they improve performance.
    \item Using a semi-supervised VAE to generate additional labelled datapoints and seeing if these help improve supervised learning 
          performance.
    \item Discovering the most important genes for phenotypes using saliency and comparing those to highly important genes found through
          other methods.  
\end{itemize}

\section{Final remarks}

I hope that the models used in this project can be refined further to find real use in the fields of medicine and biology. The models and tool
developed in this project are all freely available online and will hopefully be of use to future researchers interested in semi-supervised
learning.